%%%%%%%%%%%%%%%%%%%%%%%%%%%%%%%%%%%%%%%%%%%%%%%%%%%%%%%%%%%%%%%%%%%%%%%%%%%%%%%%%%%%%%%%%%%
%% Author: Pedro Pongelupe - Professor at PUC-MG                                         %%
%% contacts:                                                                             %%
%%    e-mail: pedro.pongelupe@sga.pucminas.br                                            %%
%%%%%%%%%%%%%%%%%%%%%%%%%%%%%%%%%%%%%%%%%%%%%%%%%%%%%%%%%%%%%%%%%%%%%%%%%%%%%%%%%%%%%%%%%%%
\documentclass{lib/eng_softdoc}

\usepackage[utf8]{inputenc}
\usepackage[portuguese]{babel}
\usepackage{tikz-uml}

%% Informations that will be insert in the table header 
\def\prof{Pedro Pongelupe Lopes}
\def\class{Programação Modular}
\def\semester{2024.1}
\def\registration{}
\def\student{}

\def\theme{Exercício}

\begin{document}
%% Table with the header
\makeheader

\problem Em uma Universidade, um aluno é cadastrado no sistema acadêmico com nome, número de matrícula e código do
curso que está realizando. Para se formar, além de cursar todas as disciplinas do currículo, o aluno precisa cumprir uma
carga de atividades complementares.

Uma atividade complementar pode ser de três tipos: formativa, extensionista ou monitoria. Cada atividade tem
uma descrição e uma carga horária. As atividades geram 1 crédito de acordo com a carga horária e seu tipo: as formativas
geram 1 crédito a cada 30h; as extensionistas, a cada 15h e as de monitoria a cada 20h. Os créditos podem ter valores
fracionários. Para se formar, o aluno precisa acumular pelo menos 4 créditos e ter participado de pelo menos 2 tipos
diferentes de atividade.

Como o sistema acadêmico já tem implementado e funcionando a parte de matrículas, disciplinas, notas e
aprovação, cabe a você planejar esta parte das atividades complementares.


\subproblem Utilizando todos os conceitos vistos até hoje na disciplina, \textbf{modele um diagrama de classes UML} para o problema proposto. O modelo deve incluir \textbf{classes, relacionamentos, atributos e métodos} necessários para resolver completamente o problema.
\textbf{Não é necessário incluir} construtores ou métodos get/set, mas indique as visibilidades de métodos e atributos. 

\subproblem Considerando seu modelo em (a), escreva o código dos métodos envolvidos na tarefa de \textbf{calcular quantos créditos foram gerados por uma atividade complementar}. Note que, dependendo do seu modelo, pode ser necessário escrever o código de mais de um método nesta questão.

\subproblem Considerando seu modelo em (a), escreva o código dos métodos envolvidos na tarefa de \textbf{verificar se um aluno já cumpriu os requisitos de atividades complementares para poder se formar}. Note que, dependendo do seu modelo, pode ser necessário escrever o código de mais de um método nesta questão.

\subproblem Utilizando sintaxe JUnit, escreva o \textbf{código de testes unitários} para seu(s) método(s) da resposta (c).

\end{document}
