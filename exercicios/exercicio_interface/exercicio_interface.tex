%%%%%%%%%%%%%%%%%%%%%%%%%%%%%%%%%%%%%%%%%%%%%%%%%%%%%%%%%%%%%%%%%%%%%%%%%%%%%%%%%%%%%%%%%%%
%% Author: Pedro Pongelupe - Professor at PUC-MG                                         %%
%% contacts:                                                                             %%
%%    e-mail: pedro.pongelupe@sga.pucminas.br                                            %%
%%%%%%%%%%%%%%%%%%%%%%%%%%%%%%%%%%%%%%%%%%%%%%%%%%%%%%%%%%%%%%%%%%%%%%%%%%%%%%%%%%%%%%%%%%%
\documentclass{lib/eng_softdoc}

\usepackage[utf8]{inputenc}
\usepackage[portuguese]{babel}
\usepackage{tikz-uml}

%% Informations that will be insert in the table header 
\def\prof{Pedro Pongelupe Lopes}
\def\class{Programação Modular}
\def\semester{2024.1}
\def\registration{}
\def\student{}

\def\theme{Exercício Interfaces}

\begin{document}
%% Table with the header
\makeheader

\problem Baseando-se na modelagem do diagrama de classes a seguir do sistema de gestão acadêmica,
complete-o para atender o novo requisito. Agora será possível alunos, professores e monitores de laboratório terem acesso às máquinas do laborátorio. Para acessar as máquinas é necessário utilizar
um usuário de login e uma senha, o usuário é a matrícula da pessoa e a senha é gerada pela seguinte 
combinação: \{CódigoPessoa\}@\{idade\}$>$\{primeiro nome\}.

Exemplo: Professor Pedro Pongelupe Lopes, código pessoa 1046730, 25 anos.
\begin{itemize}
        \item usuário: PF1046730
        \item senha: 1046730@25$>$Pedro
\end{itemize}

Complete o digrama para atender o novo requsito proposto, além disso, escreva o código e o testes 
unitários para essa nova funcionalidade.

\vspace{0.5cm}
      \begin{tikzpicture} 
         \umlclass[font=\fontsize{7}{7}\selectfont,x=-8,y=2,anchor=west,type=abstract]{PessoaUniversitaria}{--  codigoPessoa : String \\ -- nome : String \\ -- idade : int}{+ getCodigoPessoa() : string \\ + setCodigoPessoa(c: string): void \\ + getNome() : string \\ + setNome(nome: string): void \\+ getIdade() : int \\ + setIdade(idade: int): void \\ + getMatricula() : string \\ \umlvirt{\# getPrefixo(): string}}
         \umlclass[font=\fontsize{7}{7}\selectfont,x=-8, y=-4]{Aluno}{ -- materiaMatriculadas : string[]}{ + matricular(m: string) : boolean \\ \# getPrefixo() : string}
         \umlclass[font=\fontsize{7}{7}\selectfont,x=-3, y=-4]{Professor}{-- cargaHoraria : int}{ + getCargaHoraria() : int \\ \# getPrefixo() : string}
         \umlclass[font=\fontsize{7}{7}\selectfont,x=2, y=-4]{FuncionarioAdministrativo}{-- setor : string}{ + getSetor() : string \\ \# getPrefixo() : string}
         \umlVHVinherit{Aluno}{PessoaUniversitaria}
         \umlVHVinherit{Professor}{PessoaUniversitaria}
         \umlVHVinherit{FuncionarioAdministrativo}{PessoaUniversitaria}
         \umlemptyclass[font=\fontsize{7}{7}\selectfont,x=0, y=-7]{Limpeza}
         \umlemptyclass[font=\fontsize{7}{7}\selectfont,x=3, y=-7]{MonitorLaboratório}
         \umlVHVinherit{Limpeza}{FuncionarioAdministrativo}
         \umlVHVinherit{MonitorLaboratório}{FuncionarioAdministrativo}
      \end{tikzpicture}

\vspace{0.5cm}
\textbf{Observação}: Esse diagrama segue as regras que definimos em sala de aula, caso dúvidas, olhar o slide de classes abstratas que explica mais detalhadamente as regras.

\end{document}
