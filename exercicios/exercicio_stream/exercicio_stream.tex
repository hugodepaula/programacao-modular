%%%%%%%%%%%%%%%%%%%%%%%%%%%%%%%%%%%%%%%%%%%%%%%%%%%%%%%%%%%%%%%%%%%%%%%%%%%%%%%%%%%%%%%%%%%
%% Author: Pedro Pongelupe - Professor at PUC-MG                                         %%
%% contacts:                                                                             %%
%%    e-mail: pedro.pongelupe@sga.pucminas.br                                            %%
%%%%%%%%%%%%%%%%%%%%%%%%%%%%%%%%%%%%%%%%%%%%%%%%%%%%%%%%%%%%%%%%%%%%%%%%%%%%%%%%%%%%%%%%%%%
\documentclass{lib/eng_softdoc}

\usepackage{listings}
\usepackage[utf8]{inputenc}
\usepackage[portuguese]{babel}
\usepackage{tikz-uml}

%% Informations that will be insert in the table header 
\def\prof{Pedro Pongelupe Lopes}
\def\class{Programação Modular}
\def\semester{2024.1}
\def\registration{}
\def\student{}

\def\theme{Exercícios Stream}

\lstset{language=Java,
   basicstyle=\scriptsize,
   commentstyle=\color{red},
   showstringspaces=false,
   numbers=none,
   numberstyle=\tiny}
\begin{document}
%% Table with the header
\makeheader

\problem Analise o seguinte trecho de código Java:

\begin{lstlisting}
public class Livro {
   private String titulo;
   private String genero;
   private int anoPublicacao;
   private Autor autor;
}

public class Autor {
   private String nome;
   private int anoNascimento;
}
\end{lstlisting}

\textbf{Ambas as classes possuem getters e setters}. Implmente os seguintes métodos utilizando Java Stream (esses métodos não estão dentro das classes acima) :

\subproblem Dado uma lista de Livros, retorne se existe pelo menos um livro publicado em um terminado ano, ex: 1998.
\begin{lstlisting}
public boolean existeLivrosPublicadosEm(List<Livro> livros, int ano)
\end{lstlisting}
\vspace*{2cm}

\subproblem Dado uma lista de Livros, retorne todos os livros publicados de um determinado gênero, ex: Romance.
\begin{lstlisting}
public List<Livro> getLivrosPorGenero(List<Livro> livros, String genero)
\end{lstlisting}
\vspace*{2cm}


\subproblem Dado uma lista de Livros, retorne todos os livros publicados cujo o título inicie com derminada letra, ex: J.
\begin{lstlisting}
public List<Livro> getLivrosPublicadosComTitulosIniciadoEm(List<Livro> livros, char letra)
\end{lstlisting}
\vspace*{2cm}

\subproblem Dado uma lista de Livros, retorne a quantidade de livros publicados por um determinado autor, ex: Jorge Amado .
\begin{lstlisting}
public int getQuantidadeLivrosPorAutor(List<Livro> livros, String nomeAutor)
\end{lstlisting}
\vspace*{2cm}

\subproblem Dado uma lista de Livros, retorne todos os livros publicados cujo o autor tenha o publicado com uma determinada idade, ex: Livros que foram publicados quando o autor tinha 32 anos .
\begin{lstlisting}
public List<Livro> getLivrosPublicadosQuandoAutorTinhaIdade(List<Livro> livros, int idadeAutorNaPublicacao)
\end{lstlisting}

\end{document}
