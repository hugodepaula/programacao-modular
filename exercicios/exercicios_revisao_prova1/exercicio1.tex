%%%%%%%%%%%%%%%%%%%%%%%%%%%%%%%%%%%%%%%%%%%%%%%%%%%%%%%%%%%%%%%%%%%%%%%%%%%%%%%%%%%%%%%%%%%
%% Author: Pedro Pongelupe - Professor at PUC-MG                                         %%
%% contacts:                                                                             %%
%%    e-mail: pedro.pongelupe@sga.pucminas.br                                            %%
%%%%%%%%%%%%%%%%%%%%%%%%%%%%%%%%%%%%%%%%%%%%%%%%%%%%%%%%%%%%%%%%%%%%%%%%%%%%%%%%%%%%%%%%%%%
\documentclass{lib/eng_softdoc}

\usepackage[utf8]{inputenc}
\usepackage[portuguese]{babel}
\usepackage{tikz-uml}

%% Informations that will be insert in the table header 
\def\prof{Pedro Pongelupe Lopes}
\def\class{Programação Modular}
\def\semester{2024.1}
\def\registration{}
\def\student{}

\def\theme{Exercício de Revisão Prova 1}

\begin{document}
%% Table with the header
\makeheader

\problem  Uma conta bancária é identificada pelo seu número de 5 dígitos e pelo CPF de seu proprietário. As contas bancárias permitem operações de saque e depósito. A operação de saque tem uma regra especial, permitindo que o cliente faça um saque acima do seu saldo – o que é chamado de limite. Por exemplo, se uma conta tem saldo de R\$200 e limite de R\$100, seu cliente pode sacar até R\$300. O saldo fica negativo e, no próximo depósito, a conta cobrará uma taxa de 3\% sobre este valor negativo, descontando do próprio valor de depósito.

Para a situação apresentada, modele uma classe utilizando UML e implemente seu código em Java. É aconselhado testar os métodos implementados para ter certeza do seu funcionamento.

\vspace{0.5cm}

\problem Considere o seguinte diagrama UML. 

\vspace{0.5cm}
\begin{tikzpicture}
        \umlemptyclass{Autor}
        \umlemptyclass[x=4]{Livro}
        \umlemptyclass[x=8]{Leitor}
        \umlassoc[mult1=1..2, mult2=1..3]{Autor}{Livro}
        \umlassoc[mult1=1..5, mult2=0..4]{Livro}{Leitor}
\end{tikzpicture}

\vspace{0.5cm}
\subproblem Se existem 6 instâncias de autores, qual é o mínimo e o máximo de livros? E qual o mínimo e máximo de leitores?
% 6 e 18; 0 e 72
\subproblem Se existem 6 instâncias de leitores, qual é o mínimo e o máximo de livros? E qual o mínimo e máximo de autores?
% 6 e 30; 6 e 12

\vspace{0.5cm}

\problem Considere o seguinte diagrama UML 

\vspace{0.5cm}
\begin{tikzpicture}
        \umlclass{Firma}{-- nome \\ -- endereco}{}
        \umlclass[x=4]{Funcionário}{-- nome \\ -- salário}{}
        \umlaggreg[mult1=*, mult2=*]{Firma}{Funcionário}
\end{tikzpicture}

\vspace{0.5cm}
O atributo nome da classe \textit{Firma} não pode repetir. Escreva quais são as alternativas \textbf{verdadeiras}.

\subproblem Cada funcionário trabalha para pelo menos uma firma
\subproblem Cada firma tem que ter pelo menos um funcionário  
\subproblem Nenhum funcionário pode trabalhar para mais de uma firma
\subproblem Dois funcionários com o mesmo nome não podem trabalhar na mesma firma 

\vspace{0.5cm}
% a, c

\problem Qual a diferença de interpretação entre os relacionamentos livro-sobrecapa e lirvo-páginas? 

\vspace{0.5cm}
\begin{tikzpicture}
        \umlemptyclass{Livro}
        \umlclass[y=-3]{SobreCapa}{}{}
        \umlemptyclass[x=4]{Página}
        \umlemptyclass[x=8]{Parágrafo}
        \umlcompo[mult1=1, mult2=1..*]{Livro}{Página}
        \umlcompo[mult1=1, mult2=1..*]{Página}{Parágrafo}
        \umlaggreg[mult1=1, mult2=0..1]{Livro}{SobreCapa}
\end{tikzpicture}

\vspace{0.5cm}
%Livro-página é uma agregação por composição, logo se destruirmos o livro, somem as
%páginas e os parágrafos. Representamos um livro como algo fisicamente constituído de
%páginas.
%Na agregação livro-sobrecapa representamos que um livro pode ser constituído por uma
%sobrecapa mas que a sobrecapa é um objeto que existe independentemente do livro. Os
%tempos de vida dos objetos não são semelhantes.

\end{document}
