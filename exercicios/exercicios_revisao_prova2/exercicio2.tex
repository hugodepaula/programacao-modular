%%%%%%%%%%%%%%%%%%%%%%%%%%%%%%%%%%%%%%%%%%%%%%%%%%%%%%%%%%%%%%%%%%%%%%%%%%%%%%%%%%%%%%%%%%%
%% Author: Pedro Pongelupe - Professor at PUC-MG                                         %%
%% contacts:                                                                             %%
%%    e-mail: pedro.pongelupe@sga.pucminas.br                                            %%
%%%%%%%%%%%%%%%%%%%%%%%%%%%%%%%%%%%%%%%%%%%%%%%%%%%%%%%%%%%%%%%%%%%%%%%%%%%%%%%%%%%%%%%%%%%
\documentclass{lib/eng_softdoc}

\usepackage{listings}
\usepackage[utf8]{inputenc}
\usepackage[portuguese]{babel}
\usepackage{tikz-uml}

%% Informations that will be insert in the table header 
\def\prof{Pedro Pongelupe Lopes}
\def\class{Programação Modular}
\def\semester{2024.1}
\def\registration{}
\def\student{}

\def\theme{Exercício de Revisão Prova 2}

\lstset{language=Java,
   basicstyle=\scriptsize,
   commentstyle=\color{red},
   showstringspaces=false,
   numbers=none,
   numberstyle=\tiny}
\begin{document}
%% Table with the header
\makeheader

\problem Analise os seguintes fragmentos de código e responda: Houve violação de algum princípio SOLID? Se sim, qual e por quê? Reescreva o fragmento para ajustar o problema, caso necessário. 

\textbf{Para todos os problemas, todas as classes têm os getters, setters e os construtores que forem necessários.}

\subproblem
\begin{lstlisting}
public abstract class Estabelecimento  {
        private String razaoSocial;
}
public class Cafeteria extends Estabelecimento {
        public int getAreaConstruidaCafeteria() { return 70; }
        public double getValorMetroQuadradoCafeteria() { return 40d; }
        public double getAliquotaCafeteria() { return 0.5; }
}
public class Livraria extends Estabelecimento {
        public int getAreaConstruidaLivraria() { return 150; }
        public double getValorMetroQuadradoLivraria() { return 10d; }
        public double getAliquotaLivraria() { return 0.8; }
}

public class CalculadoraIPTU {

   public double calculaIPTU(Estabelecimento e) {
        double iptu = 0;

        if (e instanceOf Cafeteria) {
                Cafeteria c  = (Cafeteria) e;
                iptu = c.getAreaConstruidaCafeteria() * c.getValorMetroQuadradoCafeteria() * c.getAliquotaCafeteria();
        }
        if (e instanceOf Livraria) {
                Livraria l  = (Livraria) e;
                iptu = l.getAreaConstruidaLivraria() * l.getValorMetroQuadradoLivraria() * l.getAliquotaLivraria();
        }

        return iptu;
   }

}
\end{lstlisting}


\vspace{0.25cm}
\subproblem
\begin{lstlisting}
public abstract class InstrumentoMusical {
        private String modelo;
        private String ano;

        public abstract void tocar();

        public abstract void afinar();
}
public class Piano implements InstrumentoMusical {
        private Corda[] cordas;
        
        @Override
        public void tocar() { /*...*/ }
        
        @Override
        public void afinar() { /*...*/ }

        public void trocarCordas(Corda[] cordas) { this.cordas = cordas; }
}
public class Trompete implements InstrumentoMusical {
        private Surdina surdina;
        
        @Override
        public void tocar() { /*...*/ }
        
        @Override
        public void afinar() { /*...*/ }

        public void adicionarSurdina(Surdina s) { this.surdina = s; } 
}
\end{lstlisting}


\vspace{1cm}

\subproblem
\begin{lstlisting}
public class GerenciadorDeSistema {

        public void adicionaUsuario(Usuario u) { /*..*/ }

        public void reomveUsuario(Usuario u) { /*..*/ }

        public void enviaNotificacao(String notificacao) { /*..*/ }

        public void enviaEmail(String email, Usuario destinatario, Usuario remetente) { /*..*/ }

} 
\end{lstlisting}
\vspace{0.5cm}

\problem Para cada cenário proposto, discuta uma possível implementação utilizando uma \textit{collection} para resolver o problema.
Justifique a escolha da \textit{collection}.

\vspace{0.5cm}
\subproblem Em um sistema de gestão acadêmica, precisamos de uma classe Aluno é responsável por armazenar os dados e comportamentos dessa entidade do sistema.
Essa classe é responsável por gerir as notas, permitindo adicionar, editar, consultar e deletar dessas notas.

\vspace{0.5cm}
\subproblem Em um sistema de restaurante, precisamos de uma classe para gerenciar todas as reservas de clientes caso o restaurante não tenha mesas vagas. A ordem de chegada deve ser a mesma ordem da saída, ou seja, o primeiro a chegar deve ser o primeiro a sair.  

\vspace{0.5cm}
\subproblem Em um sistema de stream de música, precisamos de uma classe para a playlist de músicas. Uma playlist pode conter entre 0 e 150 músicas únicas,
ela pode ordernar as músicas de várias maneiras, como: ordem alfabética, duração das músicas e pela ordem de inserção. Nesse sistema, a playlist só pode ter uma ordernação e a ordenação padrão é feita pela ordem de inserção. 

\vspace{0.5cm}
\subproblem Em um sistema para a agência de transito, precisamos de uma classe para calcular o valor do IPVA de um carro.
O IPVA é calculado pela multiplicação do valor venal do veículo pela alíquota referente ao Estado qual o veículo foi registrado,
portanto, existem 27 alíquotas no Brasil.

\end{document}
