%%%%%%%%%%%%%%%%%%%%%%%%%%%%%%%%%%%%%%%%%%%%%%%%%%%%%%%%%%%%%%%%%%%%%%%%%%%%%%%%%%%%%%%%%%%
%% Author: Pedro Pongelupe - Professor at PUC-MG                                         %%
%% contacts:                                                                             %%
%%    e-mail: pedro.pongelupe@sga.pucminas.br                                            %%
%%%%%%%%%%%%%%%%%%%%%%%%%%%%%%%%%%%%%%%%%%%%%%%%%%%%%%%%%%%%%%%%%%%%%%%%%%%%%%%%%%%%%%%%%%%
\documentclass{lib/eng_softdoc}

\usepackage{listings}
\usepackage[utf8]{inputenc}
\usepackage[portuguese]{babel}
\usepackage{tikz-uml}

%% Informations that will be insert in the table header 
\def\prof{Pedro Pongelupe Lopes}
\def\class{Programação Modular}
\def\semester{2024.1}
\def\registration{}
\def\student{}

\def\theme{Exercício de Revisão Prova 3}

\lstset{language=Java,
   basicstyle=\scriptsize,
   commentstyle=\color{red},
   showstringspaces=false,
   numbers=none,
   numberstyle=\tiny}
\begin{document}
%% Table with the header
\makeheader

\problem Analise os seguintes fragmentos de código e responda: 
\begin{itemize}
        \item Esse código compila? Se não, porquê?
        \item Esse código tem alguma exceção não tratada? Se sim, qual?
        \item Se complia e não tem nenhuma exceção não tratada, qual a saída esperada no console?
\end{itemize}

\subproblem
\begin{lstlisting}
public class Driver {
	public static void main(String[] args) {
		int i = 0;
		int y = 6;

		try {
			y /= i;
			i = 5;
			y += 2;
		} catch (RuntimeException e) {
			i++;
			y = 4;
		}
		System.out.println(i + y);
	}
}
\end{lstlisting}


\vspace{0.25cm}
\subproblem
\begin{lstlisting}
public class Driver {
	public static void main(String[] args) {
		var str = "cr";
		try {
			str += "e";
		} catch (Exception e) {
			str += "i";
		} finally {
			str += "me";
		}

		System.out.println(str);
	}
}
\end{lstlisting}


\vspace{0.5cm}

\subproblem
\begin{lstlisting}
public class Driver {
	public static void main(String[] args){
		var anoAtual = LocalDate.now().get(ChronoField.YEAR);

		if (anoAtual == 1998) {
			throw new Exception("Nao estamos mais em 1998");
		} else {
			System.out.println("Estamos no ano de: " + anoAtual);
		}
	}
}
\end{lstlisting}
\vspace{0.5cm}

\subproblem
\begin{lstlisting}
public class Driver {
	public static void main(String[] args) {
		var arr = new int[] { 1, 9, 5, 0 };
		for (var i : arr) {
			if (i % 2 == 0)
				throw new NumeroParException();
		}
		System.out.println("somente numeros impares!");
	}
	public static class NumeroParException extends RuntimeException {}
}
\end{lstlisting}

\problem Considere o seguinte código Java: 

\vspace{0.5cm}
\begin{lstlisting}
public enum MesAno {
	JAN(31), FEV(29), MAR(31), ABR(30), MAI(31), JUN(30), JUL(31), AGO(31), SET(30), OUT(31), NOV(30), DEZ(31);

	private final int quantidadeDias;

	private MesAno(int quantidadeDias) {
		this.quantidadeDias = quantidadeDias;
	}
	public boolean tem31Dias() {
		return quantidadeDias == 31;
	}
	public static MesAno buscaPorNumero(int numero) {
		return MesAno.values()[numero - 1];
	}
}
public class Data {

	private int dia;

	private MesAno mes;

	private int ano;

	public Data(int dia, int mes, int ano) {
		this.dia = dia;
		this.mes = MesAno.buscaPorNumero(mes);
		this.ano = ano;
	}

	public int getDia() {
		return dia;
	}

	public void setDia(int dia) {
		this.dia = dia;
	}

	public MesAno getMes() {
		return mes;
	}

	public void setMes(MesAno mes) {
		this.mes = mes;
	}

	public int getAno() {
		return ano;
	}

	public void setAno(int ano) {
		this.ano = ano;
	}

	public boolean mesTem31Dias() {
		return mes.tem31Dias();
	}
}
\end{lstlisting}
\subproblem Desenhe o diagrama UML do código.  
\vspace{0.5cm}
\subproblem Altere o método \textit{buscaPorNumero(int numero)} para ele lançar uma exceção quando o paramêtro \textit{numero} não 
for válido. Ex: \textit{numero = 19}
\vspace{0.5cm}
\subproblem 
Implemente um método \textit{main} que instâncie as seguintes 3 datas:
\begin{itemize}
	\item 22/04/1870
	\item 30/06/2002
	\item 15/-15/2038 
\end{itemize}
\textbf{Utilizando try-catch}, caso alguma data seja inválida, instâncie \textbf{qualquer} data válida como valor padrão. 

\vspace{0.5cm}
\newpage

\problem Analise a seguinte situação proposta e o diagrama UML proposto como solução do problema :
		
		Em um cinema, há 5 salas que possuem o número da sala e 30 assentos disponíveis cada. 
    Em um dia de funcionamento desse cinema, as salas funcionam exibindo sessões, cada sessão possui um horário e o filme em cartaz.
    A bilheteria do cinema vende os ingressos das sessões ao público.
    Cada ingresso custa R\$ 20,00 a inteira e R\$ 10,00 a meia entrada que é dada para pessoas menores de 21 anos e
    estudantes. Além disso, para acompanhamento da saúde financeira do cinema, é necssário falar qual o valor arrecadado com os ingressos vendidos
    de cada sessão.

\vspace{0.5cm}
    \begin{tikzpicture}
            \umlclass{Sala}{+ \underline{QNT\_ASSENTOS}: int = 30 \\ -- numeroSala: int}{+ getNumeroSala(): int}
            \umlclass[x=8]{Sessão}{-- filme: String \\ -- horario: Date \\ -- sala: Sala \\ -- ingressosVendidos: List$<$Ingresso$>$ \\ -- quantidadeVendida: int = 0}{+ getValorArrecadado(): double \\ + venderIngresso(i : Ingresso): boolean}
            \umlassoc[mult1=1..5, mult2=1]{Sala}{Sessão}
            \umlclass[x=8, y=-4.5, type=enumeration]{Ingresso}{INTEIRA(20), MEIA(10)\\ -- preco: double}{+ getPreco(): double \\ \underline{+ ingressoPorIdadeEhEstudante(idade: int, isEstudante: boolean):Ingresso}}
            \umlcompo[mult1=1, mult2=0..30]{Sessão}{Ingresso}
    \end{tikzpicture}

\vspace{0.5cm}
		\textbf{Escreva o código} Java correspondente ao diagrama anterior.

\end{document}
