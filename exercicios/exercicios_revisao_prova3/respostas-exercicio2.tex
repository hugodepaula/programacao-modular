%%%%%%%%%%%%%%%%%%%%%%%%%%%%%%%%%%%%%%%%%%%%%%%%%%%%%%%%%%%%%%%%%%%%%%%%%%%%%%%%%%%%%%%%%%%
%% Author: Pedro Pongelupe - Professor at PUC-MG                                         %%
%% contacts:                                                                             %%
%%    e-mail: pedro.pongelupe@sga.pucminas.br                                            %%
%%%%%%%%%%%%%%%%%%%%%%%%%%%%%%%%%%%%%%%%%%%%%%%%%%%%%%%%%%%%%%%%%%%%%%%%%%%%%%%%%%%%%%%%%%%
\documentclass{lib/eng_softdoc}

\usepackage{listings}
\usepackage[utf8]{inputenc}
\usepackage[portuguese]{babel}
\usepackage{tikz-uml}

%% Informations that will be insert in the table header 
\def\prof{Pedro Pongelupe Lopes}
\def\class{Programação Modular}
\def\semester{2024.1}
\def\registration{}
\def\student{}

\def\theme{Exercício de Revisão Prova 2}

\lstset{language=Java,
        basicstyle=\scriptsize,
        commentstyle=\color{red},
        showstringspaces=false,
        numbers=none,
numberstyle=\tiny}
\begin{document}
%% Table with the header
\makeheader

\problem Analise os seguintes fragmentos de código e responda: Houve violação de algum princípio SOLID? Se sim, qual e por quê? Reescreva o fragmento para ajustar o problema, caso necessário. 

\textbf{Para todos os problemas, todas as classes têm os getters, setters e os construtores que forem necessários.}

\subproblem
\begin{lstlisting}
public abstract class Estabelecimento  {
        private String razaoSocial;
}
public class Cafeteria extends Estabelecimento {
        public int getAreaConstruidaCafeteria() { return 70; }
        public double getValorMetroQuadradoCafeteria() { return 40d; }
        public double getAliquotaCafeteria() { return 0.5; }
}
public class Livraria extends Estabelecimento {
        public int getAreaConstruidaLivraria() { return 150; }
        public double getValorMetroQuadradoLivraria() { return 10d; }
        public double getAliquotaLivraria() { return 0.8; }
}

public class CalculadoraIPTU {

   public double calculaIPTU(Estabelecimento e) {
        double iptu = 0;

        if (e instanceOf Cafeteria) {
                Cafeteria c  = (Cafeteria) e;
                iptu = c.getAreaConstruidaCafeteria() * c.getValorMetroQuadradoCafeteria() * c.getAliquotaCafeteria();
        }
        if (e instanceOf Livraria) {
                Livraria l  = (Livraria) e;
                iptu = l.getAreaConstruidaLivraria() * l.getValorMetroQuadradoLivraria() * l.getAliquotaLivraria();
        }

        return iptu;
   }

}
\end{lstlisting}
\answer O princípio representado pela letra 'O', \textit{Open-Close Principle}, foi violado, 
pois as classes \textit{Cafeteria} e \textit{Livraria} estabelecem métodos que deveriam estar na superclasse como métodos abstratos.
Consequentemente, a classe \textit{CalculadoraIPTU} deverá ser alterada para cada nova especialização de \textit{Estabelecimento}.
\begin{lstlisting}
public abstract class Estabelecimento  {
        private String razaoSocial;
        public abstract int getAreaConstruida();
        public abstract double getValorMetroQuadrado();
        public abstract double getAliquota();
}
public class Cafeteria extends Estabelecimento {
        @Override
        public int getAreaConstruida() { return 70; }
        @Override
        public double getValorMetroQuadrado() { return 40d; }
        @Override
        public double getAliquota() { return 0.5; }
}
public class Livraria extends Estabelecimento {
        @Override
        public int getAreaConstruida() { return 150; }
        @Override
        public double getValorMetroQuadrado() { return 10d; }
        @Override
        public double getAliquota() { return 0.8; }
}

public class CalculadoraIPTU {

   public double calculaIPTU(Estabelecimento e) {
        return e.getAreaConstruida() * e.getValorMetroQuadrado() * e.getAliquota();
   }

}
\end{lstlisting}

\vspace{0.25cm}
\subproblem
\begin{lstlisting}
public abstract class InstrumentoMusical {
        private String modelo;
        private String ano;

        public abstract void tocar();

        public abstract void afinar();
}
public class Piano implements InstrumentoMusical {
        private Corda[] cordas;

        @Override
        public void tocar() { /*...*/ }

        @Override
        public void afinar() { /*...*/ }

        public void trocarCordas(Corda[] cordas) { this.cordas = cordas; }
}
public class Trompete implements InstrumentoMusical {
        private Surdina surdina;

        @Override
        public void tocar() { /*...*/ }

        @Override
        public void afinar() { /*...*/ }

        public void adicionarSurdina(Surdina s) { this.surdina = s; } 
}
\end{lstlisting}

\answer Não há violação dos princípios SOLID neste fragmento.
\vspace{1cm}

\subproblem
\begin{lstlisting}
public class GerenciadorDeSistema {

        public void adicionaUsuario(Usuario u) { /*..*/ }

        public void removeUsuario(Usuario u) { /*..*/ }

        public void enviaNotificacao(String notificacao) { /*..*/ }

        public void enviaEmail(String email, Usuario destinatario, Usuario remetente) { /*..*/ }

} 
\end{lstlisting}
\answer O princípio representado pela letra 'S', \textit{Single Responsibility Principle}, 
foi violado, pois a classe \textit{GerenciadorDeSistema} possui inumeras responsabilidades, sendo uma classe deus.
Portanto, o melhor procedimento a ser feito é refatorar a classe \textit{GerenciadorDeSistema} em outras.
\begin{lstlisting}
public class GerenciadorDeUsuarios {

        public void adicionaUsuario(Usuario u) { /*..*/ }

        public void removeUsuario(Usuario u) { /*..*/ }
} 

public class GerenciadorDeComunicacoes {

        public void enviaNotificacao(String notificacao) { /*..*/ }

        public void enviaEmail(String email, Usuario destinatario, Usuario remetente) { /*..*/ }
} 
\end{lstlisting}
\vspace{0.5cm}

\problem Para cada cenário proposto, discuta uma possível implementação utilizando uma \textit{collection} para resolver o problema.
Justifique a escolha da \textit{collection}.

\vspace{0.5cm}
\subproblem Em um sistema de gestão acadêmica, precisamos de uma classe Aluno é responsável por armazenar os dados
e comportamentos dessa entidade do sistema. Essa classe é responsável por gerir as notas, permitindo adicionar, editar, consultar
e deletar dessas notas.

\answer Para o cenário descrito, a utilização de uma lista, \textit{List}, é ideal, pois, provê todas as funcionalidades
necessárias. Além disso, não há requisitos quanto a ordem de inserção ou de remoção.

\vspace{0.5cm}
\vspace{1.0cm}
\subproblem Em um sistema de restaurante, precisamos de uma classe para gerenciar todas as reservas de clientes caso o restaurante
não tenha mesas vagas. A ordem de chegada deve ser a mesma ordem da saída, ou seja, o primeiro a chegar deve ser o primeiro
a sair. \answer Para o cenário descrito, a utilização de uma Fila, \textit{Queue}, é ideal, pois, provê a inserção e a remoção
utilizando a lógica de uma fila, logo, o primeiro que chega, é o primeiro que sai.

\vspace{0.5cm}
\subproblem Em um sistema de stream de música, precisamos de uma classe para a playlist de músicas. Uma playlist pode conter entre 0 e 150 músicas únicas,
ela pode ordernar as músicas de várias maneiras, como: ordem alfabética, duração das músicas e pela ordem de inserção. Nesse
sistema, a playlist só pode ter uma ordernação e a ordenação padrão é feita pela ordem de inserção. \answer Para o cenário
descrito, a utilização de um conjuto, \textit{Set}, é ideal, pois, permite apenas itens únicos. A implementação que poderia ser utilizada é
o TreeSet que pode receber um compartor para ordernar as músicas.

\vspace{0.5cm} 
\subproblem Em um
sistema para a agência de transito, precisamos de uma classe para calcular o valor do IPVA de um carro. O IPVA é calculado pela
multiplicação do valor venal do veículo pela alíquota referente ao Estado qual o veículo foi registrado, portanto, existem 27
alíquotas no Brasil.

\answer Para o cenário descrito, a utilização de um mapa, \textit{Map}, é ideal, pois, podemos guardar todas as alíquotas por estado.
Ou seja, a chave seria o Estado e o valor seria a alíquota associada.


\end{document}
